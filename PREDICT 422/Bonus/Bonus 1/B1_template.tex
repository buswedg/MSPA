\documentclass[]{article}
\usepackage{lmodern}
\usepackage[compact]{titlesec}
\usepackage{amssymb,amsmath}
\usepackage{ifxetex,ifluatex}
\usepackage{fixltx2e} % provides \textsubscript
\ifnum 0\ifxetex 1\fi\ifluatex 1\fi=0 % if pdftex
  \usepackage[T1]{fontenc}
  \usepackage[utf8]{inputenc}
\else % if luatex or xelatex
  \ifxetex
    \usepackage{mathspec}
  \else
    \usepackage{fontspec}
  \fi
  \defaultfontfeatures{Ligatures=TeX,Scale=MatchLowercase}
\fi
% use upquote if available, for straight quotes in verbatim environments
\IfFileExists{upquote.sty}{\usepackage{upquote}}{}
% use microtype if available
\IfFileExists{microtype.sty}{%
\usepackage{microtype}
\UseMicrotypeSet[protrusion]{basicmath} % disable protrusion for tt fonts
}{}
\usepackage[margin=1in]{geometry}
\usepackage{hyperref}
\hypersetup{unicode=true,
            pdftitle={Bonus Assignment 1},
            pdfauthor={Darryl Buswell},
            pdfborder={0 0 0},
            breaklinks=true}
\urlstyle{same}  % don't use monospace font for urls
\usepackage{graphicx,grffile}
\makeatletter
\def\maxwidth{\ifdim\Gin@nat@width>\linewidth\linewidth\else\Gin@nat@width\fi}
\def\maxheight{\ifdim\Gin@nat@height>\textheight\textheight\else\Gin@nat@height\fi}
\makeatother
% Scale images if necessary, so that they will not overflow the page
% margins by default, and it is still possible to overwrite the defaults
% using explicit options in \includegraphics[width, height, ...]{}
\setkeys{Gin}{width=\maxwidth,height=\maxheight,keepaspectratio}
\IfFileExists{parskip.sty}{%
\usepackage{parskip}
}{% else
\setlength{\parindent}{0pt}
\setlength{\parskip}{6pt plus 2pt minus 1pt}
}
\setlength{\emergencystretch}{3em}  % prevent overfull lines
\providecommand{\tightlist}{%
  \setlength{\itemsep}{0pt}\setlength{\parskip}{0pt}}
\setcounter{secnumdepth}{0}
% Redefines (sub)paragraphs to behave more like sections
\ifx\paragraph\undefined\else
\let\oldparagraph\paragraph
\renewcommand{\paragraph}[1]{\oldparagraph{#1}\mbox{}}
\fi
\ifx\subparagraph\undefined\else
\let\oldsubparagraph\subparagraph
\renewcommand{\subparagraph}[1]{\oldsubparagraph{#1}\mbox{}}
\fi

%%% Use protect on footnotes to avoid problems with footnotes in titles
\let\rmarkdownfootnote\footnote%
\def\footnote{\protect\rmarkdownfootnote}

%%% Change title format to be more compact
\usepackage{titling}

% Create subtitle command for use in maketitle
\newcommand{\subtitle}[1]{
  \posttitle{
    \begin{center}\large#1\end{center}
    }
}

\setlength{\droptitle}{-2em}
  \title{Bonus Assignment 1}
  \pretitle{\vspace{\droptitle}\centering\huge}
  \posttitle{\par}
\subtitle{MSPA PREDICT 422-DL-56 LEC}
  \author{Darryl Buswell}
  \preauthor{\centering\large\emph}
  \postauthor{\par}
  \date{}
  \predate{}\postdate{}

\begin{document}
\maketitle

\section{1 Introduction}\label{introduction}

This document presents results of the first bonus assessment for the
Masters of Science in Predictive Analytics course: PREDICT 422. This
assessment required the student to perform cluster-wise regression for
automobile data from the rpart package.

\section{2 Bonus Assessment}\label{bonus-assessment}

\subsection{2.1 Load the Dataset}\label{load-the-dataset}

As a first step, we load the `car.test.frame' data from the rpart
package and observe the structure of the data.

We can see that the dataset includes a number of numeric and factor type
variables:

\begin{itemize}
\tightlist
\item
  Price: a numeric vector giving the list price in US dollars of a
  standard model
\item
  Country: of origin, a factor with levels France, Germany, Japan ,
  Japan/USA, Korea, Mexico, Sweden and USA
\item
  Reliability: a numeric vector coded 1 to 5
\item
  Mileage: fuel consumption miles per US gallon, as tested.
\item
  Type: a factor with levels Compact Large Medium Small Sporty Van
\item
  Weight: kerb weight in pounds
\item
  Disp.: the engine capacity (displacement) in litres
\item
  HP: the net horsepower of the vehicle.
\end{itemize}

`Price' will be our response variable, whilst `Mileage', `Weight',
`Disp', `HP' will form our set of predictor variables.

\subsection{2.2 Model Fit: Single
Cluster}\label{model-fit-single-cluster}

For this assessment, we leverage the FlexMix package in order to perform
cluster-wise regression. FlexMix implements a general framework for
finite mixtures of regression models, with parameter estimation being
performed using the EM algorithm. We first fit a cluster-wise regression
using a single cluster (the full set of observations). The model fit is
shown below.

\begin{verbatim}
## 
## Call:
## flexmix(formula = Price ~ Mileage + Weight + Disp. + HP, 
##     data = car.test.frame, k = 1)
## 
##        prior size post>0 ratio
## Comp.1     1   60     60     1
## 
## 'log Lik.' -555.404 (df=6)
## AIC: 1122.808   BIC: 1135.374
\end{verbatim}

We can also observe the parameter estimates over the cluster by using
the `parameters' function.

\begin{verbatim}
##                       Comp.1
## coef.(Intercept)  822.181804
## coef.Mileage     -165.400112
## coef.Weight         4.602914
## coef.Disp.        -40.567694
## coef.HP            70.908074
## sigma            2642.448365
\end{verbatim}

We note the polarity of the majority of coefficients estimate seems
reasonable, with an increase in fuel consumption translating to a
decrease in price, or an increase in horse power resulting in an
increase in price. The weight and displacement variables are more
difficult to justify however.

\subsection{2.3 Model Fit: Two Clusters}\label{model-fit-two-clusters}

We can repeat this exercise by refitting the cluster-wise regression
with two clusters. The model fit is shown below.

\begin{verbatim}
## 
## Call:
## flexmix(formula = Price ~ Mileage + Weight + Disp. + HP, 
##     data = car.test.frame, k = 2)
## 
##        prior size post>0 ratio
## Comp.1   0.7   46     55 0.836
## Comp.2   0.3   14     55 0.255
## 
## 'log Lik.' -538.7509 (df=13)
## AIC: 1103.502   BIC: 1130.728
\end{verbatim}

We can see that the dataset has been broken into two samples of 46 and
14 observations in length. We note the improvement in AIC and BIC values
reported for the above model versus the single cluster model previously
fit.

Again, we can also observe the parameter estimates over both clusters by
using the `parameters' function.

\begin{verbatim}
##                       Comp.1
## coef.(Intercept) 6101.909096
## coef.Mileage     -193.960149
## coef.Weight         3.561308
## coef.Disp.        -12.628352
## coef.HP            13.304388
## sigma            1173.502924
\end{verbatim}

\begin{verbatim}
##                      Comp.2
## coef.(Intercept) 2867.77610
## coef.Mileage     -463.78457
## coef.Weight        11.91847
## coef.Disp.        -91.74390
## coef.HP            27.43067
## sigma            2151.60300
\end{verbatim}

We can see that although the polarity of coefficient is consistent over
both clusters, the magnitude of those coefficients does have some
variation.


\end{document}
